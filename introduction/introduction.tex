\chapter{Introduction}
\label{introduction}
The centralization of power on Internet platforms raises up issues of fairness %cause unfairness in many industries. 
The technological advancements of the Internet has mainly lead to income polarization. The explosive increase in revenue on Internet platforms has mainly lead to higher profit margins for Big Tech %instead of 
at the expense of the core contributors on these platforms. During the last decades, we observe the trend of \textit{platformization}: a concentration of economic activity %from happening on a wide range of companies to 
with a few major platforms run by Big Tech corporations. 
% Add a few numbers, or a graph showing hard facts on the rise of platformization
This trend has driven %is highly susceptible to 
the rise of monopolies and oligarchs.  %This results in small and untraceable payouts to the core contributors on such platforms. 
The users of centralized platforms are often negatively affected by the decisions driven by the logic of business models rather than democratic considerations. 
%that are made based on business strategies instead of democratic procedures.

Platformization already has a strong effect on the music industry, %in which the 
wher few succesful music streaming services have an immense amount of power. The biggest streaming services dictate the rules which artists have to play by. The top 5 streaming services and the top 3 labels dominate the market.
% Add exact percentages of market domination
Independent artists often have a hard %time to 
making a living because the streaming companies and labels take large revenue cuts of up to 40\%. The processing of royalties through many intermediaries is unclear on purpose. Artists can also suffer from political interference %in 
from these companies, reducing their freedom of speech. This thesis %aims to distribute 
proposes a solution how to re-distribute the power from centralized streaming platforms to listeners and artists. The goals of our approach are twofold: %with the goals that artists 
(1) ensure that artist obtain a fair income and; (2) that they are freed from censorship. 
\\
\\
The main contributions of this thesis are:
\begin{enumerate}
    \item A novel framework as an alternative for Big Tech: the robot economy in software;
    \item A partial implementation of this framework: a fully decentralized Android music streaming application \textit{MusicDAO} which attempts to liberate artists and consumers from powerful intermediaries.
\end{enumerate}
\\
MusicDAO implements a few key components of our framework: P2P leaderless infrastructure, resilient communication, trustless content sharing/exploration and a trustless monetary system (see \ref{tab:robot-economy-building-blocks}). We design, implement and evaluate the MusicDAO system. We perform supervised experiments with 10 Android devices to test the responsiveness and latency of its main features. Aditionally, we release the application publicly to 50+ users and measure the monetary flows from users to artists over time. 
% is 50+ correct?
It is available on Google Play\footnote{\url{https://play.google.com/store/apps/details?id=nl.tudelft.trustchain}} and its code is open source\footnote{\url{github.com/Tribler/trustchain-superapp/}}.

% Add key findings from Evaluation

% https://www.opendemocracy.net/en/can-europe-make-it/robot-economy-full-automation-work-future/

\section{Monopolization on the Internet}
The Internet is moving from a network serving the people to a sparse selection of platforms controlling nearly all e-commerce. 
%over which nearly all commerce is regulated. 
The consumer choice is diminishing due to the power of oligarchs and monopolies known as 'Big Tech'. 
%A few Big Tech corporations are gaining increasing power in the the surface on which market exchange takes place. 
These corporations take a large cut in revenue streams which majorly affects income for creators of content and services. Examples are Uber, %Uber is not a content platform?  
Ebay and Spotify. Revenue streams towards the core contributors on platforms are opaque on purpose. The software and algorithms powering these platforms are also a black box to creators and consumers. Furthermore, users have negligible influence on the rules of these platforms. 
%future of these platforms. 
As explained by \citep{stiglitz2019market}, the fundamental problem is the growing ``concentration of market power, which allows dominant firms to exploit their customers and squeeze their employees, whose own bargaining power and legal protections are being weakened'', while ``[...] CEOs and executives are extracting higher pay for themselves''.

\section{Towards a robot economy}
An alternative for Big Tech economy is %building 
a robot economy in software. Our novel vision of a robot economy follows recent theoretical groundwork by \cite{arduengo2020robot}: In a robot economy, intelligent robots play a key role, by performing economic operations autonomously.\footnote{%Explain that by robots you mean not actual robots but algorithms}  
Robot tasks are driven by artificial intelligence, and cooperate with humans. While robots already take an active part in society today, the key difference in this vision is that robots have internal funds (which may be money, tokens or other assets) and can perform transactions on their own. Our work sets the first steps towards robot economy in software. Software as a robot economy is a service that runs autonomously, with which humans interact. Humans can spend funds, perform decisions or interact with data, while the software is run by robots. 

A robot economy in software can have large influence: it can replace a company, or even a full value chain, by software. With this vision, we can design and implement systems that work in favor of its users instead of a company. In a traditional software system, a company decides the parameters and functions of software. The software is run on company infrastructure only, which creates central points of failure. In a software system in the robot economy, its parameters and functions are voted on by its user base in a democratic voting process. It is run on distributed infrastructure, so is not susceptible to central points of failure. Since to the introduction of blockchains and crypto-currency, this voting process can be automated and transparent. Robots make intelligent decisions based on data and value given by humans. Such an implementation can replace entire value chains of industries.We argue that recent developments in autonomous software systems such as Decentralized Autonomous Organizations (DAOs) provide feasible solutions for such an approach \cite{%example} 


\begin{figure}
    \centering
    \includegraphics[width=1\textwidth]{introduction/robot-economy.png}
    \caption{A robot economy in software: democratic and autonomous software}
    \label{fig:robot-economy-in-software}
\end{figure}

A full robot economy aims to have the following key characteristics: 
\begin{itemize}
    \item Automated;
    \item Transparent;
    \item Fair;
    \item Democratic;
    \item Open (permissionless);
    \item Leaderless;
    \item Self-evolving.
\end{itemize}
To accomplish these key characteristics in a software system, we envision the main building blocks to be as described in fig. \ref{tab:robot-economy-building-blocks}.

\section{Centralization of power in the music industry}
An industry %with great consequences of this trend 
emblematic of the technologically driven centralization trend is the music industry. In the last 20 years there has been a remarkably fast shift from the exchange of CDs in various stores to music streaming on the Internet. Music platforms and labels use their economic muscle to push down artist salaries. They take large cuts of revenue from the user subscription money.

Firstly, corporations with power squeeze the music production side by taking large cuts of revenue from the user subscription money. As a result, the artists receive a low compensation. Especially independent artists have a hard time making a living. The distributors Spotify, iTunes and Google Play take on a 25\% to 40\% revenue cut.

Secondly, Big Tech has curatorial power to decide what is shown in the catalog of their application. The music catalog may seem endless, but in reality it is controlled by the Big Tech corporation and dictated by the interests of major labels. The inner workings of recommendation algorithms and playlists are in the hands of a few labels and streaming services.

Finally, the streaming companies can censor content creators. The freedom of artist expression is curtailed by the undemocratic judgments. Essentially, Big Tech has the power to decide the future of an artist.

\begin{figure}
    \centering
	\includegraphics[width=0.4\textwidth]{introduction/problem-image.png}
	\caption{Artist compensation inconsistency}
\end{figure}

\section{Proposed solution: MusicDAO}
This thesis proposes an alternative technology from Big Tech streaming services. We observe that most functions of music streaming systems are already completely automated.
% Add a number: how much % is already automated?
We design and implement a decentralized system which attempts to replace the full value chain in music streaming industry, from the subscription money to the artist, by removing all intermediaries and giving power back to the artists and listeners. Listeners can stream music without being dependent on a single provider and can give money directly to artists. Artists receive 100\% of this donation and subscription money. %Are there any transaction fees?

In essence, the solution is a decentralized autonomous organization (DAO) which is formed by listeners and artists. A DAO is defined by \cite{buterin2014dao} as an ``entity that lives on the internet and exists autonomously, but also heavily relies on hiring individuals to perform certain tasks that the automation itself cannot do'' (see \ref{fig:dao-quadrants}). In this thesis we present the design and implementation of our mobile android app MusicDAO: a music streaming service for the common good. Users of this app form a phone-to-phone zero-server network over which they publish music, download music and transfer money. This proof-of-principle of a DAO shows a fair and transparent music streaming service in which no external servers, third parties or intermediaries are necessary. Any person can join the network, publish music and get paid for it. 100\% of music revenue goes to artists.

MusicDAO supports the following functionalities:
\begin{itemize}
    \item Defining and publishing music content with metadata;
    \item Streaming music over BitTorrent;
    \item Caching and streaming optimization algorithms;
    \item Browsing playlists;
    \item Remote keyword search;
    \item Peer-to-peer donations to artists using Bitcoin.
\end{itemize}

\subsection{Experimentation and evaluation}
In a real-world experiment with Android phones, we tested the feasibility of such a phone-to-phone infrastructure-less system. We ran MusicDAO on at least X android devices, and registered the latency of retrieving music metadata, and transaction speeds of transferring money and audio files. 
% Add key findings (but longer than in intro-intro)

\begin{table}[]
\begin{tabular}{|l|l|}
\hline
\textbf{Component}                         & \textbf{Focus in MusicDAO} \\ \hline
Peer-to-peer leaderless infrastructure     & \checkmark                                      \\ \hline
Resilient communication                    & \checkmark                                      \\ \hline
Trustless content sharing and exploration  & \checkmark                                      \\ \hline
Trustless monetary system                  & \checkmark                                      \\ \hline
Democratic user engagement                      & $\times$                                      \\ \hline
AI for decision making (robot tasks)       & $\times$                                      \\ \hline
Continuous code evolution and distribution & $\times$                                       \\ \hline
\end{tabular}
\caption{The main components to achieve a robot economy in software}
\label{tab:robot-economy-building-blocks}
\end{table}
% Repeat this table in the Design section

% Most important: How can artists distribute and sell their work in a digital economy beholden to ruthlessly commercial and centralized interests?
% https://thebaffler.com/salvos/the-problem-with-muzak-pelly
